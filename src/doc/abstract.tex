%%%%%%%%%%%%%%%%%%%%%%%%%%%%%%%%%%%%%%%%%%%%%%%%%%%%%%%%%%%%%%%%%%%%%%%%%%%%%%%%

\begin{abstract}
\thispagestyle{empty}
O obxectivo último deste proxecto foi a creación dunha gaita MIDI baseada en
software e hardware libre, que tivo coma obxecto a simulación o máis fielmente
posible dunha gaita galega, así como a recolleita e realización das esixencias
do usuario dado que, a día de hoxe, non existe ningunha gaita MIDI no mercado
que cumpra coas mesmas na súa totalidade.\\

O proxecto comezou analizando as gaitas MIDI xa existentes no mercado,
recollendo as opinións tanto de expertos coma dos usuarios xerais, para logo
plasmalos nun deseño que, finalmente, nos permitiu implementar o controlador
MIDI correspondente.\\

Para iso fíxose uso da plataforma Arduino (hardware libre modular), empregando
tanto placas base (Uno), coma módulos externos (sensores de presión e
capacitivos) ou superpostos que se comunican entre si empregando protocolos moi
variados segundo as súas necesidades (I2C, UART, USB, MIDI, etc.). Todo este
hardware está programado en C/C++, Processing/Wiring e Arduino.\\

Para o control de dito hardware e do sintetizador faise uso dunha aplicación de
configuración gráfica multiplataforma escrita en Java e Swing, que explota toda
a potencionalidade dos mesmos, facendo que a simulación sexa o máis completa
posible.\\

\textbf{Palabras clave}: Gaita galega; Controlador MIDI; Microprogramación;
Redes sen fíos; Tempo real; Hardware libre; Software libre; Arduino; C/C++;
Zigbee; Java; JSON.

\end{abstract}

%%%%%%%%%%%%%%%%%%%%%%%%%%%%%%%%%%%%%%%%%%%%%%%%%%%%%%%%%%%%%%%%%%%%%%%%%%%%%%%%
