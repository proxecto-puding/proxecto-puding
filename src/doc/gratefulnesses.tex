\thispagestyle{empty}
\section*{Agradecementos}

É de ben nacido ser agradecido e son moitos os agradecementos que debería
plasmar nesta memoria; pero para ilo faríame falta un anexo completo. Sendo o
espazo do que dispoño máis ben reducido, gustaríame aproveitalo para dar as
grazas a aquelas persoas máis especiais.

\begin{itemize}
 \item A \textit{Tiago M. Fernández}, por titorizar e dirixir este proxecto,
       así como pola súa inestimable e nunca ben pagada axuda cando as cousas
       se torceron.
 \item A \textit{Santiago J. Barro}, por dirixir este proxecto e animarme a
       plasmar nel a idea na que levaba máis dunha década cavilando.
 \item A \textit{Lis Latas}, pola súa axuda á hora de construir o corpo do
       punteiro que acompaña estas páxinas.
 \item A \textit{Pepe Vaamonde}, por prestarse coma músico de estudo para
       gravar as mostras de son que emprega esta gaita.
 \item A \textit{Óscar Rodríguez Fernández}, por ensinarme practicamente todo
       aquilo que sei sobre o mundo da gaita galega.
 \item A \textit{Jacobo Aragunde} e \textit{Javier Morán}, pola axuda prestada
       con \textit{LibrePlan}.
 \item A \textit{Laura M. Castro}, por proporcionar desinteresadamente o seu
       modelo para a memoria do proxecto feito en \LaTeX.
 \item A todas aquelas persoas anónimas que se molestaron en perder parte do
       seu tempo para cubrir a enquisa do estudo de viabilidade que acompaña
       esta memoria.
\end{itemize}

A eles e a todos aqueles que se me quedan no tinteiro, os meus máis sinceiros
agradecementos.
