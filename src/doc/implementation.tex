\chapter{Desenvolvemento do sistema}
\minitoc
\label{chap:implementacion}
\vspace{0.5cm}

%%%%%%%%%%%%%%%%%%%%%%%%%%%%%%%%%%%%%%%%%%%%%%%%%%%%%%%%%%%%%%%%%%%%%%%%%%%%%%%%
% Objetivo: Exponer las partes relevantes de la implementación                 %
%%%%%%%%%%%%%%%%%%%%%%%%%%%%%%%%%%%%%%%%%%%%%%%%%%%%%%%%%%%%%%%%%%%%%%%%%%%%%%%%

\lettrine{N}{este} capítulo exporase o desenvolvemento do proxecto baseándose
no esquema proporcionado pola planificación inicial: desde o deseño software e
hardware de baixo nivel do sistema ata a implementación e o ensamblado do
producto, pasando por un prototipo operacional.

\section{Determinación}

 \subsection{Obxectivos}

 Establecéronse os obxectivos da fase de desenvolvemento do proxecto. \\

 Obxectivos:

 \begin{itemize}
  \item Implementar unha gaita MIDI sen fíos en tempo real empregando
        software/hardware libre.
 \end{itemize}

 \subsection{Alternativas}

 Establecéronse posibles alternativas a eses obxectivos, aplicables no caso de
 que estes non se puidesen cumprir. \\

 Alternativas:

 \begin{itemize}
  \item Se non é posible implementar completamente o proxecto, pode optarse
        por:
        \begin{enumerate}
         \item Se é por causa de que o hardware non o permite, cambiar as pezas
               correspondentes por outras que si o permitan.
         \item Se as partes que non é posible implementar son opcionais ou de
               pouco peso, desbotalas e implementar o resto.
         \item Cancelar e mudar de proxecto.
        \end{enumerate}
 \end{itemize}

 \subsection{Restriccións}

 Establecéronse restriccións aplicables a ditos obxectivos.

 \begin{enumerate}
  \item As propias restriccións veñen dadas polo propio título do proxecto. A
        saber:
        \begin{enumerate}
         \item Empregar o protocolo MIDI.
         \item Empregar tecnoloxía sen fíos.
         \item Empregar tempo real.
         \item Empregar software libre.
         \item Empregar harwdware libre.
         \item E/ou as derivadas de calquera das súas alternativas.
        \end{enumerate}
 \end{enumerate}

\section{Avaliación de alternativas e resolución de riscos}

 \subsection{Análise de riscos}

 Determináronse os riscos que comportaban as distintas alternativas e as súas
 posibles solucións.

 \begin{enumerate}
  \item Alternativas 1.
        \begin{enumerate}
         \item Riscos:
               \begin{enumerate}
                \item Que non exista hardware alternativo que soporte a
                      implemtación das características restantes.
                \item Que as partes a desbotar sexan partes importantes ou
                      incluso críticas.
                \item Que o tempo restante para a execución do proxecto non
                      sexa suficiente.
               \end{enumerate}
         \item Solucións:
               \begin{enumerate}
                \item Recortar características ou cancelar e mudar de proxecto.
                \item Aplicar medidas de mitigación para que a planificación
                      non se vexa afectada en extremo, ou cancelar e mudar de
                      proxecto se fan inviable o mesmo.
                \item Agardar a presentalo na seguinte convocatoria.
               \end{enumerate}
        \end{enumerate}
 \end{enumerate}

 \subsection{Prototipo operacional}

  \subsubsection{Prototipo hardware}

   \paragraph{Integración do hardware}

   \paragraph{Encapsulamento do hardware}

  \subsubsection{Prototipo software}

   \paragraph{Desenvolvemento do prototipo}

   \paragraph{Gravación das mostras}

\section{Desenvolvemento e validación do seguinte nivel do producto}

 \subsection{Simulacións, modelos e programas de proba}

 \subsection{Deseño detallado}

  \subsubsection{Deseño hardware}

  \subsubsection{Deseño software}

 \subsection{Ensamblado e codificación}

  \subsubsection{Ensamblado}

  \subsubsection{Codificación}

 \subsection{Probas de unidade}

 \subsection{Integración e probas}

 \subsection{Probas de aceptación}

 \subsection{Implantación}