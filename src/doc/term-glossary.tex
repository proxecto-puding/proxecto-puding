\chapter{Glosario de termos}
\label{chap:glosario-terminos}

%%%%%%%%%%%%%%%%%%%%%%%%%%%%%%%%%%%%%%%%%%%%%%%%%%%%%%%%%%%%%%%%%%%%%%%%%%%%%%%%
% Objetivo: Lista de términos utilizados en el documento,                      %
%           junto con sus respectivos significados.                            %
%%%%%%%%%%%%%%%%%%%%%%%%%%%%%%%%%%%%%%%%%%%%%%%%%%%%%%%%%%%%%%%%%%%%%%%%%%%%%%%%

\begin{description}
 \item [Bordón] Tubo sonoro que produce unha nota pedal.
 \item [Chillón] Tubo sonoro que sae do fol paralelo á ronqueta, que soa na
       dominante da tonalidade do punteiro e na mesma oitava.
 \item [Controlador MIDI] Xerador de mensaxes MIDI.
 \item [Dixitación] Maneira concreta de colocación dos dedos sobre o punteiro.
 \item [Nota pedal] Na música tonal, unha nota pedal é unha nota sostida ou
       continua que se mantén mentres o resto da melodía segue avanzando.
 \item [Fol] Bolsa onde se almacena o aire que fai soar a gaita.
 \item [Gaita] Instrumento musical de vento que consta dun fol e dun ou máis
       tubos sonoros.
 \item [Punteiro] Tubo cónico que sae polo frontal do fol e que é o que produce
       a melodía principal en base á colocación dos dedos.
 \item [Ronco] Tubo sonoro que se apoia sobre o ombro do gaiteiro, que soa na
       mesma tonalidade que o punteiro, pero dúas oitavas máis grave.
 \item [Ronqueta] Tubo sonoro que se apoia sobre o antebrazo dereito do
       gaiteiro, que soa na mesma tonalidade que o punteiro, pero unha oitava
       máis grave.
 \item [Secuenciador MIDI] Gravador de mensaxes MIDI.
 \item [Sintetizador MIDI] Xerador de son a partir de mensaxes MIDI.
\end{description}
